%% -*- coding: utf-8 -*-
\newpage

\chapter{Conclusions}
\label{chap:conclusions}

This chapter presents the concluding remarks about our work. Discuss our contributions in Section~\ref{sec:contributions}, and present directions for future works in Section~\ref{sec:future-work}.

In this thesis, we presented a novel graphical approach to scenario reduction on time series ensembles. We evaluated the feasibility of our proposal by performing an empirical study with a series of potential users, both experienced in the area and not. By observing their interactions and interviewing them afterwards, we obtained valuable insights on the usefulness of our proposal. Following from the results of our previous publication we have expanded our work in two aspects: (i)~use glyph sizes to represent the time in the Time-lapsed LAMP chart; (ii)~in the same chart, encode the uncertainty inherent to the data. The first expansion was done in order to fix an issue related to the abstraction created by employing multidimensional-projections using time-varying data. The second expansion aims to help users to quickly identify time ranges with high variance in the data.

Besides the user study, we also compared our results with other approaches in the literature and industry. proposes to select a set of representative scenarios under different well configurations. Their approach handles several simulation parameters and responses simultaneously, thus, selecting representative scenarios based on multiple criteria. The Industry approach selects scenarios with cumulative production closest to the target references at the end of the simulation. Both approaches are well suited depending on the post-processing tasks and the data itself. In our tests, our approach selected scenarios with consistently smaller error when compared to both the Industry and Clustering approaches. It must be stated, however, that this does not mean that our approach is better, only that it yields good scenarios when the objective function is the proximity to a reference in a context where the scenarios' evolution must be taken into consideration.


\section{Publications}
\label{sec:publications}

\section{Contributions}
\label{sec:contributions}

\section{Future Work}
\label{sec:future-work}
